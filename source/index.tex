% Options for packages loaded elsewhere
% Options for packages loaded elsewhere
\PassOptionsToPackage{unicode}{hyperref}
\PassOptionsToPackage{hyphens}{url}
\PassOptionsToPackage{dvipsnames,svgnames,x11names}{xcolor}
%
\documentclass[
  russian,
  12pt,
  a4paper,
]{article}
\usepackage{xcolor}
\usepackage[top=2.5cm,bottom=2.5cm,left=2.5cm,right=2.5cm]{geometry}
\usepackage{amsmath,amssymb}
\setcounter{secnumdepth}{5}
\usepackage{iftex}
\ifPDFTeX
  \usepackage[T1]{fontenc}
  \usepackage[utf8]{inputenc}
  \usepackage{textcomp} % provide euro and other symbols
\else % if luatex or xetex
  \usepackage{unicode-math} % this also loads fontspec
  \defaultfontfeatures{Scale=MatchLowercase}
  \defaultfontfeatures[\rmfamily]{Ligatures=TeX,Scale=1}
\fi
\usepackage{lmodern}
\ifPDFTeX\else
  % xetex/luatex font selection
\fi
% Use upquote if available, for straight quotes in verbatim environments
\IfFileExists{upquote.sty}{\usepackage{upquote}}{}
\IfFileExists{microtype.sty}{% use microtype if available
  \usepackage[]{microtype}
  \UseMicrotypeSet[protrusion]{basicmath} % disable protrusion for tt fonts
}{}
\usepackage{setspace}
\makeatletter
\@ifundefined{KOMAClassName}{% if non-KOMA class
  \IfFileExists{parskip.sty}{%
    \usepackage{parskip}
  }{% else
    \setlength{\parindent}{0pt}
    \setlength{\parskip}{6pt plus 2pt minus 1pt}}
}{% if KOMA class
  \KOMAoptions{parskip=half}}
\makeatother
% Make \paragraph and \subparagraph free-standing
\makeatletter
\ifx\paragraph\undefined\else
  \let\oldparagraph\paragraph
  \renewcommand{\paragraph}{
    \@ifstar
      \xxxParagraphStar
      \xxxParagraphNoStar
  }
  \newcommand{\xxxParagraphStar}[1]{\oldparagraph*{#1}\mbox{}}
  \newcommand{\xxxParagraphNoStar}[1]{\oldparagraph{#1}\mbox{}}
\fi
\ifx\subparagraph\undefined\else
  \let\oldsubparagraph\subparagraph
  \renewcommand{\subparagraph}{
    \@ifstar
      \xxxSubParagraphStar
      \xxxSubParagraphNoStar
  }
  \newcommand{\xxxSubParagraphStar}[1]{\oldsubparagraph*{#1}\mbox{}}
  \newcommand{\xxxSubParagraphNoStar}[1]{\oldsubparagraph{#1}\mbox{}}
\fi
\makeatother


\usepackage{longtable,booktabs,array}
\usepackage{calc} % for calculating minipage widths
% Correct order of tables after \paragraph or \subparagraph
\usepackage{etoolbox}
\makeatletter
\patchcmd\longtable{\par}{\if@noskipsec\mbox{}\fi\par}{}{}
\makeatother
% Allow footnotes in longtable head/foot
\IfFileExists{footnotehyper.sty}{\usepackage{footnotehyper}}{\usepackage{footnote}}
\makesavenoteenv{longtable}
\usepackage{graphicx}
\makeatletter
\newsavebox\pandoc@box
\newcommand*\pandocbounded[1]{% scales image to fit in text height/width
  \sbox\pandoc@box{#1}%
  \Gscale@div\@tempa{\textheight}{\dimexpr\ht\pandoc@box+\dp\pandoc@box\relax}%
  \Gscale@div\@tempb{\linewidth}{\wd\pandoc@box}%
  \ifdim\@tempb\p@<\@tempa\p@\let\@tempa\@tempb\fi% select the smaller of both
  \ifdim\@tempa\p@<\p@\scalebox{\@tempa}{\usebox\pandoc@box}%
  \else\usebox{\pandoc@box}%
  \fi%
}
% Set default figure placement to htbp
\def\fps@figure{htbp}
\makeatother



\ifLuaTeX
\usepackage[bidi=basic,provide=*]{babel}
\else
\usepackage[bidi=default,provide=*]{babel}
\fi
% get rid of language-specific shorthands (see #6817):
\let\LanguageShortHands\languageshorthands
\def\languageshorthands#1{}


\setlength{\emergencystretch}{3em} % prevent overfull lines

\providecommand{\tightlist}{%
  \setlength{\itemsep}{0pt}\setlength{\parskip}{0pt}}



 


\makeatletter
\@ifpackageloaded{tcolorbox}{}{\usepackage[skins,breakable]{tcolorbox}}
\@ifpackageloaded{fontawesome5}{}{\usepackage{fontawesome5}}
\definecolor{quarto-callout-color}{HTML}{909090}
\definecolor{quarto-callout-note-color}{HTML}{0758E5}
\definecolor{quarto-callout-important-color}{HTML}{CC1914}
\definecolor{quarto-callout-warning-color}{HTML}{EB9113}
\definecolor{quarto-callout-tip-color}{HTML}{00A047}
\definecolor{quarto-callout-caution-color}{HTML}{FC5300}
\definecolor{quarto-callout-color-frame}{HTML}{acacac}
\definecolor{quarto-callout-note-color-frame}{HTML}{4582ec}
\definecolor{quarto-callout-important-color-frame}{HTML}{d9534f}
\definecolor{quarto-callout-warning-color-frame}{HTML}{f0ad4e}
\definecolor{quarto-callout-tip-color-frame}{HTML}{02b875}
\definecolor{quarto-callout-caution-color-frame}{HTML}{fd7e14}
\makeatother
\makeatletter
\@ifpackageloaded{bookmark}{}{\usepackage{bookmark}}
\makeatother
\makeatletter
\@ifpackageloaded{caption}{}{\usepackage{caption}}
\AtBeginDocument{%
\ifdefined\contentsname
  \renewcommand*\contentsname{Содержание}
\else
  \newcommand\contentsname{Содержание}
\fi
\ifdefined\listfigurename
  \renewcommand*\listfigurename{Список Иллюстраций}
\else
  \newcommand\listfigurename{Список Иллюстраций}
\fi
\ifdefined\listtablename
  \renewcommand*\listtablename{Список Таблиц}
\else
  \newcommand\listtablename{Список Таблиц}
\fi
\ifdefined\figurename
  \renewcommand*\figurename{Рисунок}
\else
  \newcommand\figurename{Рисунок}
\fi
\ifdefined\tablename
  \renewcommand*\tablename{Таблица}
\else
  \newcommand\tablename{Таблица}
\fi
}
\@ifpackageloaded{float}{}{\usepackage{float}}
\floatstyle{ruled}
\@ifundefined{c@chapter}{\newfloat{codelisting}{h}{lop}}{\newfloat{codelisting}{h}{lop}[chapter]}
\floatname{codelisting}{Список}
\newcommand*\listoflistings{\listof{codelisting}{Список Каталогов}}
\makeatother
\makeatletter
\makeatother
\makeatletter
\@ifpackageloaded{caption}{}{\usepackage{caption}}
\@ifpackageloaded{subcaption}{}{\usepackage{subcaption}}
\makeatother
\usepackage{bookmark}
\IfFileExists{xurl.sty}{\usepackage{xurl}}{} % add URL line breaks if available
\urlstyle{same}
\hypersetup{
  pdftitle={База знаний СНК кафедры фармакологии Российского университета медицины},
  pdfauthor={Сергей Царегородцев},
  pdflang={ru},
  colorlinks=true,
  linkcolor={blue},
  filecolor={Maroon},
  citecolor={blue},
  urlcolor={blue},
  pdfcreator={LaTeX via pandoc}}


\title{База знаний СНК кафедры фармакологии Российского университета
медицины}
\usepackage{etoolbox}
\makeatletter
\providecommand{\subtitle}[1]{% add subtitle to \maketitle
  \apptocmd{\@title}{\par {\large #1 \par}}{}{}
}
\makeatother
\subtitle{Электронный учебник по научной работе для студентов-медиков}
\author{Сергей Царегородцев}
\date{2025-07-28}
\begin{document}
\maketitle

\renewcommand*\contentsname{Содержание}
{
\hypersetup{linkcolor=}
\setcounter{tocdepth}{2}
\tableofcontents
}

\setstretch{1.5}
\bookmarksetup{startatroot}

\chapter{База знаний СНК кафедры
фармакологии}\label{ux431ux430ux437ux430-ux437ux43dux430ux43dux438ux439-ux441ux43dux43a-ux43aux430ux444ux435ux434ux440ux44b-ux444ux430ux440ux43cux430ux43aux43eux43bux43eux433ux438ux438}

Электронный учебник по научной работе для студентов-медиков

\hfill\break

\bookmarksetup{startatroot}

\chapter{О проекте}\label{sec-about}

\begin{figure}[H]

{\centering \includegraphics[width=3.125in,height=\textheight,keepaspectratio]{cover.png}

}

\caption{СНК кафедры фармакологии Российского университета медицины}

\end{figure}%

Добро пожаловать в базу знаний научного кружка кафедры фармакологии
Российского университета медицины!

Этот электронный учебник поможет вам освоить начальные навыки научной
работы:

\begin{itemize}
\tightlist
\item
  \textbf{Критическая оценка источников} - как оценивать качество
  научных статей
\item
  \textbf{Оформление библиографии} - правильное цитирование и оформление
  ссылок
\item
  \textbf{Участие в конкурсах} - актуальные научные мероприятия и
  конкурсы
\item
  \textbf{Дополнительные ресурсы} - полезные инструменты и материалы
\item
  \textbf{Сообщения из Telegram} - пересылаемые материалы из группы
  кружка
\end{itemize}

\begin{tcolorbox}[enhanced jigsaw, breakable, toprule=.15mm, colback=white, opacityback=0, rightrule=.15mm, leftrule=.75mm, arc=.35mm, bottomrule=.15mm, colframe=quarto-callout-note-color-frame, left=2mm]

\vspace{-3mm}\textbf{Связь с автором}\vspace{3mm}

\begin{itemize}
\tightlist
\item
  \textbf{Группа кружка в Telegram:}
  \href{https://t.me/pharmRUM}{@pharmRUM}
\item
  \textbf{Email:}
  \href{mailto:sergiotsar@ya.ru}{\nolinkurl{sergiotsar@ya.ru}}
\item
  \textbf{ORCID:}
  \href{https://orcid.org/0000-0002-0254-0516}{0000-0002-0254-0516}
\end{itemize}

\end{tcolorbox}

\section{Структура
учебника}\label{ux441ux442ux440ux443ux43aux442ux443ux440ux430-ux443ux447ux435ux431ux43dux438ux43aux430}

Материал организован в форме электронного учебника, который можно:

\begin{itemize}
\tightlist
\item
  Читать онлайн на сайте
\item
  Скачать в формате PDF
\item
  Использовать как справочник
\end{itemize}

База знаний постоянно обновляется, поэтому следите за обновлениями!

\section{Как
использовать}\label{ux43aux430ux43a-ux438ux441ux43fux43eux43bux44cux437ux43eux432ux430ux442ux44c}

\begin{enumerate}
\def\labelenumi{\arabic{enumi}.}
\tightlist
\item
  \textbf{Начните с раздела ``Введение''} - общие принципы научной
  работы
\item
  \textbf{Изучите поиск информации} - практические советы по работе с
  базами данных
\item
  \textbf{Освойте критическую оценку} - как анализировать качество
  исследований
\item
  \textbf{Изучите оформление} - правила цитирования и библиографии
\item
  \textbf{Узнайте о конкурсах} - актуальные возможности для студентов
\item
  \textbf{Изучите ресурсы} - дополнительные материалы и инструменты
\item
  \textbf{Следите за Telegram} - актуальные материалы из группы кружка
\end{enumerate}

\begin{tcolorbox}[enhanced jigsaw, breakable, toprule=.15mm, colback=white, opacityback=0, rightrule=.15mm, leftrule=.75mm, arc=.35mm, bottomrule=.15mm, colframe=quarto-callout-tip-color-frame, left=2mm]

\vspace{-3mm}\textbf{Совет}\vspace{3mm}

Рекомендуется изучать материал последовательно, начиная с введения и
постепенно переходя к более сложным темам.

\end{tcolorbox}

\part{Разделы}

\begin{itemize}
\item
  В главе ``Глава~\ref{sec-literature_search}'' описывается подготовка к
  поиску литературы, формулирование исследовательского вопроса, выбор
  баз данных и стратегий поиска с акцентом на доказательность.
\item
  В главе ``Глава~\ref{sec-evaluation}'' подробно рассматриваются
  критерии отбора научных публикаций, наукометрические показатели,
  квартили журналов и способы выявления потенциальных проблем и
  несоответствий в публикации (например, конфликт интересов).
\item
  В главе ``Глава~\ref{sec-literature_review}'' описывается, как
  подготовить качественный обзор литературы, избегая простого
  перечисления источников, и приводится структура научной статьи по
  модели IMRAD.
\item
  В главе ``Глава~\ref{sec-bibliography}'' даются рекомендации по
  использованию библиографических менеджеров и приведены ссылки на
  полезные инструкции и плагины. Рассматриваются требования ГОСТ для
  оформления списка литературы.
\item
  В главе ``Глава~\ref{sec-competitions}'' описываются актуальные
  конкурсы университетов и студенческого научного общества РосУниМеда,
  включая сроки подачи работ, контактную информацию и полезные ссылки.
\item
  В главе ``Глава~\ref{sec-resources}'' собраны ссылки на сайты, курсы,
  сервисы визуализации данных и другие материалы, которые могут
  пригодиться при подготовке научной работы.
\end{itemize}

Перед тем как приступить к использованию ресурса, рекомендуется
посмотреть презентацию и видеоурок, которые приведены в разделе
«Оформление библиографии и Zotero». Они помогут быстрее ориентироваться
в инструментах и требованиях к оформлению.

\begin{tcolorbox}[enhanced jigsaw, breakable, toprule=.15mm, colback=white, opacityback=0, rightrule=.15mm, leftrule=.75mm, arc=.35mm, bottomrule=.15mm, colframe=quarto-callout-caution-color-frame, left=2mm]

\vspace{-3mm}\textbf{️Все рекомендации носят информационный характер. Перед началом работы с
конкретным журналом или конференцией обязательно ознакомьтесь с
официальными требованиями к оформлению и сроками подачи материалов.}\vspace{3mm}

\end{tcolorbox}

\chapter{Поиск
литературы}\label{ux43fux43eux438ux441ux43a-ux43bux438ux442ux435ux440ux430ux442ux443ux440ux44b}

\chapter{Поиск литературы}\label{sec-literature_search}

\section{Подготовительный
этап}\label{ux43fux43eux434ux433ux43eux442ux43eux432ux438ux442ux435ux43bux44cux43dux44bux439-ux44dux442ux430ux43f}

Перед тем как погружаться в специализированные базы данных, стоит
укрепить базовые знания по теме. Начните с учебников, обзоров и
клинических рекомендаций. Учебная литература помогает получить
представление о важнейших понятиях и избежать повторения уже проведённых
исследований. Также полезно обратиться к англоязычной Википедии и
национальным клиническим рекомендациям, чтобы сформировать общий
контекст изучаемой проблемы.

\textbf{Формулирование вопроса.} Чётко сформулированный
исследовательский вопрос --- основа любого успешного поиска. Для
клинических исследований часто используют подход PICO (Population,
Intervention, Comparator, Outcomes), который помогает определить
ключевые параметры будущего исследования. Сформулируйте краткое описание
предполагаемого исследования и продумайте структуру будущей публикации.

\section{Ресурсы для
поиска}\label{ux440ux435ux441ux443ux440ux441ux44b-ux434ux43bux44f-ux43fux43eux438ux441ux43aux430}

Используйте как российские, так и международные базы данных. В России
это eLIBRARY.ru и CyberLeninka. Для международного поиска --- PubMed,
Scopus, Google Scholar, Web of Science и специализированные регистры
клинических исследований. Согласно обзору методов составления обзора
Goodfellow отмечает, что выбор корректных баз данных и использование
булевых операторов существенно повышают точность поиска.

В документах Cochrane Handbook рекомендуют обращаться не только к
стандартным библиотечным базам (MEDLINE, Embase), но и к регистрам
рандомизированных исследований, реестрам клинических испытаний и отчётам
регуляторных органов. Поиск должен быть максимально чувствительным, что
приводит к большому количеству результатов, поэтому важно использовать
фильтры и сочетать ключевые слова с контролируемыми рубрикаторами
(например MeSH).

\section{Стратегия
поиска}\label{ux441ux442ux440ux430ux442ux435ux433ux438ux44f-ux43fux43eux438ux441ux43aux430}

\begin{enumerate}
\def\labelenumi{\arabic{enumi}.}
\tightlist
\item
  \textbf{Определите ключевые понятия.} Используйте свободный текст и
  термины контролируемого словаря (MeSH, Emtree). Внутри каждого
  концепта комбинируйте синонимы с помощью оператора \texttt{OR};
  различные концепты объединяйте через \texttt{AND}.
\item
  \textbf{Установите временные рамки и типы исследований.} Для многих
  тем достаточно последних 5--10 лет, однако для исторического анализа
  может потребоваться больший интервал.
\item
  \textbf{Используйте фильтры.} Применяйте валидированные фильтры для
  конкретных типов исследований, например высокочувствительные стратегии
  Cochrane для поиска рандомизированных контролируемых исследований.
\item
  \textbf{Подключите специалиста по информационному поиску.} Авторы
  Cochrane подчёркивают, что опытный библиотекарь или информационный
  специалист может существенно улучшить поиск и снизить риск пропустить
  важные исследования.
\end{enumerate}

\section{Документирование
процесса}\label{ux434ux43eux43aux443ux43cux435ux43dux442ux438ux440ux43eux432ux430ux43dux438ux435-ux43fux440ux43eux446ux435ux441ux441ux430}

Ведение отчёта о поиске необходимо для прозрачности и воспроизводимости.
Записывайте, какие базы данных и фильтры использовались, даты последнего
поиска, количество найденных публикаций и причину исключения каждой из
них. Многие журналы требуют предоставления полного отчёта о поиске в
приложении к статье. В разделе «Оформление библиографии» описаны
инструменты, которые помогают собирать и хранить найденные источники.

\chapter{Оценка качества
статей}\label{ux43eux446ux435ux43dux43aux430-ux43aux430ux447ux435ux441ux442ux432ux430-ux441ux442ux430ux442ux435ux439}

\chapter{Принципы критической оценки}\label{sec-evaluation}

Даже хорошо выстроенный поиск возвращает сотни результатов, поэтому
необходимо установить критерии отбора. Обратите внимание на следующие
аспекты:

\begin{itemize}
\item
  \textbf{Соответствие исследовательскому вопросу.} Отбирайте только те
  публикации, которые напрямую отвечают на ваш вопрос. В обзоре
  Goodfellow подчёркивается, что анализ найденной литературы позволяет
  осознать значимость исследовательского вопроса и сформулировать
  аргументацию будущей работы.
\item
  \textbf{Актуальность.} Для большинства тем рекомендуется отдавать
  предпочтение публикациям последних 5--7~лет. Старые работы могут быть
  полезны для исторического контекста, но современные исследования дают
  наиболее точные данные о текущей практике.
\item
  \textbf{Качество методологии.} Оценивайте дизайн исследования, объём
  выборки, наличие контрольных групп, прозрачность описания методов и
  статистического анализа. Критически осмыслите ограничения, указанные
  авторами.
\item
  \textbf{Конфликт интересов и этика.} Проверьте, указаны ли источники
  финансирования, наличие заявления об этическом одобрении и доступность
  данных.
\item
  \textbf{Наукометрические показатели.} Квартильность журнала (Q1--Q4)
  помогает оценить престиж и цитируемость издания. Информацию о квартиле
  можно найти на сайте Scimago Journal \& Country Rank для международных
  журналов и в РИНЦ --- для российских. Чем выше квартиль (Q1), тем
  выше, как правило, качество рецензирования.
\item
  \textbf{Количество цитирований.} Статьи, которые часто цитируются,
  обычно оказывают заметное влияние на область, однако большое
  количество цитирований не всегда означает высокое качество (например,
  в случае критикуемых работ). Сравнивайте количество цитирований
  относительно возраста статьи и тематики.
\end{itemize}

\chapter{Практические
рекомендации}\label{ux43fux440ux430ux43aux442ux438ux447ux435ux441ux43aux438ux435-ux440ux435ux43aux43eux43cux435ux43dux434ux430ux446ux438ux438}

\begin{enumerate}
\def\labelenumi{\arabic{enumi}.}
\tightlist
\item
  \textbf{Используйте чек-листы.} Многие библиотеки предлагают готовые
  чек‑листы для критической оценки исследований (например, у Cochrane
  есть инструменты для оценки риска смещения в рандомизированных
  контролируемых исследованиях). Такие чек‑листы позволяют
  структурировать анализ.
\item
  \textbf{Сопоставляйте несколько источников.} Старайтесь найти
  систематические обзоры или метаанализы по теме --- они обычно
  объединяют данные из множества исследований и дают более взвешенные
  выводы.
\item
  \textbf{Записывайте результаты оценки.} Ведение таблицы с краткими
  выводами об уровне доказательности каждой статьи поможет позже при
  написании разделов «Обзор литературы» и «Обсуждение».
\end{enumerate}

Критическая оценка требует времени, но именно этот этап позволяет
выстроить научно обоснованную работу и избежать ссылок на недостоверные
источники.

\chapter{Написание обзора литературы и структура
IMRAD}\label{ux43dux430ux43fux438ux441ux430ux43dux438ux435-ux43eux431ux437ux43eux440ux430-ux43bux438ux442ux435ux440ux430ux442ux443ux440ux44b-ux438-ux441ux442ux440ux443ux43aux442ux443ux440ux430-imrad}

\chapter{Обзор литературы: цель и
организация}\label{sec-literature_review}

Обзор литературы~--- это аналитический раздел, который демонстрирует
понимание текущих исследований в выбранной области. Наиболее
распространённая ошибка при написании обзора --- превращать его в
последовательный реферат статей: сначала обсуждать одну работу, затем
другую, не связывая их между собой. Такой подход не синтезирует знания и
не позволяет показать связи между результатами разных авторов. Вместо
этого следует организовать обзор по темам и вопросам, которые
обсуждаются в источниках.

\section{Шаги по подготовке обзора
литературы}\label{ux448ux430ux433ux438-ux43fux43e-ux43fux43eux434ux433ux43eux442ux43eux432ux43aux435-ux43eux431ux437ux43eux440ux430-ux43bux438ux442ux435ux440ux430ux442ux443ux440ux44b}

\begin{enumerate}
\def\labelenumi{\arabic{enumi}.}
\tightlist
\item
  \textbf{Сформулируйте вопрос.} Определите исследовательский вопрос или
  основную идею, на которой будет строиться обзор. Чёткая постановка
  вопроса помогает выбрать релевантные источники.
\item
  \textbf{Соберите и критически прочитайте литературу.} Найдите
  публикации по теме и оцените их цели, основные тезисы и выводы;
  проанализируйте, насколько убедительны аргументы и как они соотносятся
  с вашим вопросом.
\item
  \textbf{Выделите ключевые материалы.} Отметьте фрагменты, которые
  напрямую связаны с вашей проблемой. Вы можете выделять текст,
  создавать таблицу или делать пометки.
\item
  \textbf{Определите общие темы и подходы.} Сгруппируйте источники по
  повторяющимся темам или методам. Выявите, какие точки зрения и подходы
  существуют, где они сходятся и расходятся.
\item
  \textbf{Составьте план.} Для каждой темы определите, какие источники
  её освещают. Определите последовательность тем, чтобы рассказ был
  логичным и строился от общих вопросов к частным.
\item
  \textbf{Напишите разделы.} Каждый абзац начинайте с тезиса (точки),
  который вы хотите донести, а затем приводите результаты
  соответствующих исследований.
\item
  \textbf{Оформляйте ссылки.} Корректно цитируйте работы в соответствии
  с требованиями вашего стиля (ГОСТ, APA, Vancouver
  и~т.п.)【188591234370092†L371-L372】.
\end{enumerate}

Такой подход помогает структурировать материалы и показать читателю,
какие аспекты темы уже хорошо исследованы, а где существуют пробелы.
Чёткое обозначение исследовательского вопроса и грамотное объединение
источников по темам делают обзор убедительным и логичным.

\chapter{Структура
IMRAD}\label{ux441ux442ux440ux443ux43aux442ux443ux440ux430-imrad}

Для представления результатов собственных исследований широко
используется формат IMRAD: Introduction (введение), Methods (методы),
Results (результаты) и Discussion (обсуждение). Эта структура
применяется для отчётов о спланированных исследованиях в медицине,
естественных и общественных науках【148792055107481†L268-L273】.

\begin{itemize}
\tightlist
\item
  \textbf{Введение (Introduction).} В первом разделе автор объясняет,
  почему исследование важно. Начните с описания проблемы, затем обзора
  текущего состояния науки; обозначьте существующий «разрыв» в знаниях и
  покажите, как ваше исследование его заполняет. Если в работе
  выдвигаются гипотезы, их следует указать в конце
  введения【148792055107481†L275-L282】.
\item
  \textbf{Методы (Methods).} Этот раздел описывает, как было проведено
  исследование: популяцию, выборку, используемые методы и оборудование.
  Методы пишут в прошедшем времени, подробно, чтобы читатель мог
  воспроизвести эксперимент【148792055107481†L284-L290】.
\item
  \textbf{Результаты (Results).} Здесь представлены лишь результаты, без
  интерпретации. Таблицы и рисунки должны быть пронумерованы, подписи
  располагают над таблицами и под
  рисунками【148792055107481†L293-L299】.
\item
  \textbf{Обсуждение (Discussion).} В обсуждении суммируют основные
  результаты, сравнивают их с другими работами, оценивают гипотезы,
  обсуждают ограничения и предлагают направления для будущих
  исследований【148792055107481†L301-L306】.
\end{itemize}

Структура IMRAD помогает читателю быстро ориентироваться в научном
тексте и понимать, как был проведён эксперимент и каковы его выводы. При
подготовке обзора литературы полезно ориентироваться на структуру
будущей статьи, чтобы понять, в какой части исследования могут быть
пробелы, и корректно запланировать собственное исследование.

\chapter{Оформление библиографии и
Zotero}\label{ux43eux444ux43eux440ux43cux43bux435ux43dux438ux435-ux431ux438ux431ux43bux438ux43eux433ux440ux430ux444ux438ux438-ux438-zotero}

\chapter{Зачем нужна правильная библиография}\label{sec-bibliography}

Правильное оформление списка литературы необходимо не только для
соответствия требованиям журнала или конференции. Корректное цитирование
чужих идей демонстрирует уважение к предыдущим исследованиям и повышает
научную репутацию автора. В обзорной статье о цитировании
подчёркивается, что точное указание источников является обязательным для
достижения научной достоверности, а использование программ для
управления ссылками помогает быстро находить, сохранять и оформлять
источники【26716079035544†L134-L137】.

\chapter{ГОСТ
Р~7.0.100--2018}\label{ux433ux43eux441ux442-ux440-7.0.1002018}

В России список литературы чаще всего оформляют по ГОСТ~Р~7.0.100--2018.
Структура ссылок зависит от типа источника (статья, книга,
интернет‑ресурс и~т.д.), но в общем виде содержит фамилию и инициалы
автора, название, выходные данные и цифровой идентификатор (DOI или
URL). Пример:

\begin{verbatim}
Munblit A. et al. Stop COVID Cohort: An Observational Study of 3480 Patients… – Clinical Infectious Diseases. – 2021. – Vol. 73, № 1. – P. 1–11. – DOI: 10.1093/cid/ciaa1535.
\end{verbatim}

При ссылке на веб‑страницу укажите URL и дату обращения. Вся литература
должна быть пронумерована в порядке появления в тексте. Для курсовых и
дипломных работ следует заранее уточнить требования кафедры или журнала.

\chapter{Зачем нужен библиографический
менеджер}\label{ux437ux430ux447ux435ux43c-ux43dux443ux436ux435ux43d-ux431ux438ux431ux43bux438ux43eux433ux440ux430ux444ux438ux447ux435ux441ux43aux438ux439-ux43cux435ux43dux435ux434ux436ux435ux440}

Использовать ручное расставление ссылок можно лишь при небольшом
количестве источников, однако это занимает много времени и увеличивает
вероятность ошибок. Библиографические менеджеры позволяют собирать,
сортировать и автоматически вставлять ссылки в документ. Среди
распространённых программ --- Zotero, JabRef, Mendeley, EndNote. На
официальном сайте Zotero отмечено, что инструмент помогает собирать,
организовывать, аннотировать, цитировать и делиться
исследованиями【461619430794529†L20-L25】; организует библиотеку с
коллекциями и тегами【461619430794529†L47-L51】; поддерживает генерацию
ссылок для текстовых редакторов и более 9~000 стилей
оформления【461619430794529†L55-L60】; обеспечивает синхронизацию данных
между устройствами и позволяет работать
совместно【461619430794529†L64-L76】.

Преимущества использования Zotero:

\begin{itemize}
\tightlist
\item
  \textbf{Сбор информации одним кликом.} После установки расширения в
  браузер можно сохранять записи из PubMed, Scopus или веб‑страницы.
\item
  \textbf{Организация и поиск.} Элементы библиотеки можно сортировать по
  коллекциям, назначать теги и оставлять аннотации. Возможен поиск по
  любому полю.
\item
  \textbf{Автоматическое цитирование.} Zotero вставляет ссылки в Word,
  LibreOffice или Google Docs и автоматически формирует список
  литературы в нужном стиле. При изменении порядка ссылок в тексте
  номера автоматически обновляются.
\item
  \textbf{Синхронизация и совместная работа.} Можно настроить
  синхронизацию через сервера Zotero или WebDAV и делиться библиотеками
  с коллегами.
\end{itemize}

\chapter{Установка и настройка
Zotero}\label{ux443ux441ux442ux430ux43dux43eux432ux43aux430-ux438-ux43dux430ux441ux442ux440ux43eux439ux43aux430-zotero}

\begin{enumerate}
\def\labelenumi{\arabic{enumi}.}
\tightlist
\item
  \textbf{Скачайте программу} с официального сайта
  \url{https://www.zotero.org} и установите расширение для браузера
  (Chrome, Firefox или Safari).
\item
  \textbf{Подключите стили цитирования.} На GitHub доступны файлы стилей
  для российских стандартов. Например, стиль
  \texttt{gost-r-7-0-100-2018.csl} можно загрузить на странице проекта
  Zotero. После скачивания файл нужно импортировать в настройки Zotero.
\item
  \textbf{Настройте синхронизацию.} Создайте бесплатный аккаунт Zotero
  или используйте сторонний сервер WebDAV. Сервисы Koofr и Mail.ru
  позволяют бесплатно хранить вложения.
\item
  \textbf{Установите плагины.} Популярные расширения включают: Translate
  for Zotero (перевод аннотаций), Sci-Hub Downloader (автоматическое
  скачивание полнотекстовых статей), Zutilo (расширенные функции
  управления) и~др.
\end{enumerate}

\chapter{Автоматическое вставление ссылок в
Word}\label{ux430ux432ux442ux43eux43cux430ux442ux438ux447ux435ux441ux43aux43eux435-ux432ux441ux442ux430ux432ux43bux435ux43dux438ux435-ux441ux441ux44bux43bux43eux43a-ux432-word}

После установки Zotero и плагина для Word можно выбирать стиль
оформления (например, ГОСТ) и вставлять цитаты с помощью пункта
«Add/Edit Citation». При вставлении новой ссылки список литературы
автоматически обновляется. Для создания перекрёстных ссылок и списка
сокращений пользуйтесь инструментами Word (см. подробный видеоурок,
предоставленный в рамках кружка).

\begin{quote}
\textbf{Совет:} перед отправкой работы в журнал проверьте, что выбранный
стиль соответствует требованиям конкретного издания. Некоторые журналы
используют модифицированные версии ГОСТ или международные стандарты,
такие как Vancouver или APA.
\end{quote}

\section{Пошаговое использование Zotero для
цитирования}\label{ux43fux43eux448ux430ux433ux43eux432ux43eux435-ux438ux441ux43fux43eux43bux44cux437ux43eux432ux430ux43dux438ux435-zotero-ux434ux43bux44f-ux446ux438ux442ux438ux440ux43eux432ux430ux43dux438ux44f}

В этой части собран пошаговый алгоритм работы с Zotero~--- от сбора
источников до формирования итогового списка литературы. Материал основан
на инструкции по автоматическому созданию списка литературы
【952735408626081†L69-L76】.

\subsection{Шаг~1. Соберите все
источники}\label{ux448ux430ux433-1.-ux441ux43eux431ux435ux440ux438ux442ux435-ux432ux441ux435-ux438ux441ux442ux43eux447ux43dux438ux43aux438}

\begin{itemize}
\tightlist
\item
  \textbf{Установите Zotero и браузерный коннектор.} Программа доступна
  на Windows, macOS и Linux; подключаемый модуль (Zotero Connector)
  можно установить для Chrome, Firefox, Safari и
  Edge【952735408626081†L69-L76】.
\item
  \textbf{Сохраняйте найденные публикации.} Чтобы впоследствии быстро
  вставлять цитаты, сначала соберите всю литературу в Zotero.
  Пользуйтесь коллекциями для распределения источников по темам и не
  беспокойтесь о стиле оформления на этом
  этапе【952735408626081†L103-L113】.
\end{itemize}

\subsection{Шаг~2. Добавьте
источники}\label{ux448ux430ux433-2.-ux434ux43eux431ux430ux432ux44cux442ux435-ux438ux441ux442ux43eux447ux43dux438ux43aux438}

\begin{itemize}
\tightlist
\item
  \textbf{Через браузерный коннектор.} Откройте интересующую страницу
  (например, статью PubMed или PDF) и нажмите на кнопку Zotero Connector
  --- программа сохранит запись вместе с метаданными и
  вложением【952735408626081†L124-L150】. При сохранении сайтов Zotero
  фиксирует дату обращения и URL.
\item
  \textbf{Через окно программы.} В Zotero нажмите зелёный значок с
  крестиком и выберите тип материала (книга, статья, веб‑страница
  и~т.д.), затем заполните поля вручную. Можно использовать кнопку в
  виде «волшебной палочки»: введите DOI, ISBN или другой идентификатор,
  чтобы Zotero автоматически подгрузил библиографию; проверяйте
  корректность данных и при необходимости исправляйте
  их【952735408626081†L153-L195】.
\end{itemize}

\subsection{Шаг~3. Читайте и
аннотируйте}\label{ux448ux430ux433-3.-ux447ux438ux442ux430ux439ux442ux435-ux438-ux430ux43dux43dux43eux442ux438ux440ux443ux439ux442ux435}

Встроенный просмотрщик PDF позволяет выделять текст и добавлять
комментарии; заметки сохраняются в базе и могут быть вставлены в
документ вместе с цитатой【952735408626081†L213-L239】.

\subsection{Шаг~4. Установите стиль
цитирования}\label{ux448ux430ux433-4.-ux443ux441ux442ux430ux43dux43eux432ux438ux442ux435-ux441ux442ux438ux43bux44c-ux446ux438ux442ux438ux440ux43eux432ux430ux43dux438ux44f}

Стиль ГОСТ не входит в стандартный набор Zotero. Загрузите файл
\texttt{gost-r-7-0-100-2018.csl} с GitHub и импортируйте его через
«Settings~→~Citation~→~+~Add Style»【952735408626081†L243-L259】. При
необходимости выберите другой стиль на сайте CitationStyles.org.

\subsection{Шаг~5. Подключите текстовый
редактор}\label{ux448ux430ux433-5.-ux43fux43eux434ux43aux43bux44eux447ux438ux442ux435-ux442ux435ux43aux441ux442ux43eux432ux44bux439-ux440ux435ux434ux430ux43aux442ux43eux440}

\begin{itemize}
\tightlist
\item
  \textbf{Google~Docs.} После установки Zotero Connector панель Zotero
  автоматически появится в меню Docs, и никакой дополнительной настройки
  не требуется【952735408626081†L276-L297】.
\item
  \textbf{Microsoft~Word и LibreOffice.} Перейдите в Zotero →
  «Settings~→~Cite~→~Word Processors» и установите интеграцию для
  выбранного редактора. Следуйте подсказкам мастера установки
  【952735408626081†L276-L297】.
\end{itemize}

\subsection{Шаг~6. Вставляйте
цитаты}\label{ux448ux430ux433-6.-ux432ux441ux442ux430ux432ux43bux44fux439ux442ux435-ux446ux438ux442ux430ux442ux44b}

В текстовом редакторе нажмите кнопку «Add/edit citation». Появится
красная строка поиска, в которой можно вводить название источника или
автора. Выберите нужную публикацию, при необходимости укажите номер
страницы, разделённый запятой или «p.». Система автоматически
пронумерует ссылку; при изменении порядка цитат в тексте номера
обновятся【952735408626081†L309-L353】.

\subsection{Шаг~7. Сформируйте список
литературы}\label{ux448ux430ux433-7.-ux441ux444ux43eux440ux43cux438ux440ux443ux439ux442ux435-ux441ux43fux438ux441ux43eux43a-ux43bux438ux442ux435ux440ux430ux442ux443ux440ux44b}

Поставьте курсор в конце документа и нажмите «Add/edit bibliography».
Zotero автоматически создаст список литературы в соответствии с
выбранным стилем и обновит его при добавлении новых
ссылок【952735408626081†L379-L386】.

\subsection{Шаг~8. Проверьте и отредактируйте
список}\label{ux448ux430ux433-8.-ux43fux440ux43eux432ux435ux440ux44cux442ux435-ux438-ux43eux442ux440ux435ux434ux430ux43aux442ux438ux440ux443ux439ux442ux435-ux441ux43fux438ux441ux43eux43a}

Автоматическая генерация не избавляет от ошибок: иногда могут
отсутствовать авторы или быть лишние поля. Проверьте полученный список и
при необходимости внесите корректировки. По опыту пользователей Word и
LibreOffice работают стабильнее, чем
Google~Docs【952735408626081†L394-L403】.

Следуя этим шагам, можно значительно ускорить подготовку курсовых и
дипломных работ, а также минимизировать вероятность ошибок в оформлении
ссылок и списка литературы.

\chapter{Конкурсы и
мероприятия}\label{ux43aux43eux43dux43aux443ux440ux441ux44b-ux438-ux43cux435ux440ux43eux43fux440ux438ux44fux442ux438ux44f}

\chapter{Основные мероприятия студенческого научного
общества}\label{sec-competitions}

\section{«Медицинская
весна~--~2025»}\label{ux43cux435ux434ux438ux446ux438ux43dux441ux43aux430ux44f-ux432ux435ux441ux43dux430-2025}

II~Международный молодёжный научный форум «Медицинская весна» пройдёт
21--23~мая~2025~года на базе Сеченовского университета. Отборочный этап
завершится 25~марта~2025~года, а результаты заочного отбора будут
объявлены 20~апреля~2025~года【851172738796405†L28-L34】. Формат форума
включает итоговую научно‑практическую студенческую конференцию, конкурс
клинических случаев «Сложный пациент» и тематические секции по
фундаментальной и клинической медицине【851172738796405†L48-L90】.
Участие бесплатное, публикация тезисов осуществляется в сборнике,
индексируемом в~РИНЦ. Подробная информация и регистрация доступны на
сайте форума.

\subsection{Конкурс клинических случаев «Сложный
пациент»}\label{ux43aux43eux43dux43aux443ux440ux441-ux43aux43bux438ux43dux438ux447ux435ux441ux43aux438ux445-ux441ux43bux443ux447ux430ux435ux432-ux441ux43bux43eux436ux43dux44bux439-ux43fux430ux446ux438ux435ux43dux442}

Участникам предлагается представить описания отдельных клинических
случаев. Допускаются студенты бакалавриата, специалитета, магистратуры и
ординаторы. Секции конкурса включают терапию, хирургию, онкологию,
акушерство и гинекологию, неврологию и
психиатрию【851172738796405†L123-L140】. Публикация тезисов и участие
бесплатны; приглашения на очный этап рассылаются
20~апреля~2025~года【851172738796405†L142-L150】.

\subsection{Конференция
«\#мояНаука»}\label{ux43aux43eux43dux444ux435ux440ux435ux43dux446ux438ux44f-ux43cux43eux44fux43dux430ux443ux43aux430}

Площадка для первого шага в науку предназначена для первокурсников
Сеченовского университета. Приём работ включает направления: гистология,
анатомия, патологическая физиология, биохимия и~др. Результаты заочного
отбора будут объявлены 20~апреля~2025~года; участие также
бесплатно【851172738796405†L168-L195】.

\subsection{«Аспирантские
чтения»}\label{ux430ux441ux43fux438ux440ux430ux43dux442ux441ux43aux438ux435-ux447ux442ux435ux43dux438ux44f}

Конференция для молодых учёных и аспирантов. Приём тезисов охватывает
клиническую медицину, биологические науки, профилактическую медицину,
психологию и педагогические направления【851172738796405†L203-L218】.
Дата проведения очного этапа совпадает с форумом «Медицинская весна»;
подробности --- в информационном письме оргкомитета.

\section{Итоговая студенческая научная конференция
(ИСНК)}\label{ux438ux442ux43eux433ux43eux432ux430ux44f-ux441ux442ux443ux434ux435ux43dux447ux435ux441ux43aux430ux44f-ux43dux430ux443ux447ux43dux430ux44f-ux43aux43eux43dux444ux435ux440ux435ux43dux446ux438ux44f-ux438ux441ux43dux43a}

Итоговая студенческая научная конференция (ИСНК) проводится в Российском
национальном исследовательском медицинском университете
имени~Н.И.~Пирогова и других вузах. Конференция предназначена для
студентов всех курсов, заинтересованных в подготовке и защите
научно‑исследовательских работ. Мероприятие обычно проходит в несколько
этапов:

\begin{enumerate}
\def\labelenumi{\arabic{enumi}.}
\tightlist
\item
  \textbf{Внутрикафедральный отбор.} Студенты представляют тезисы на
  кафедру; лучшие работы направляются на факультетский этап.
\item
  \textbf{Факультетский этап.} Секции собирают доклады по направлениям;
  жюри выбирает лучших для участия в университетском финале.
\item
  \textbf{Университетский финал.} На этом этапе проходит защита
  исследований перед экспертной комиссией. Победители получают дипломы и
  рекомендации к публикации.
\end{enumerate}

Точные даты ИСНК ежегодно меняются; следите за объявлениями на
официальном сайте университета или студенческого научного общества. Как
правило, отбор тезисов проходит осенью, а финальная конференция ---
весной (апрель--май).

\section{Конкурс студенческих научных
рефератов}\label{ux43aux43eux43dux43aux443ux440ux441-ux441ux442ux443ux434ux435ux43dux447ux435ux441ux43aux438ux445-ux43dux430ux443ux447ux43dux44bux445-ux440ux435ux444ux435ux440ux430ux442ux43eux432}

Конкурс реферативных работ предназначен для студентов младших курсов
(1--3~курс) и направлен на развитие навыков поиска литературы и
оформления обзоров. Участники пишут реферат на одну из предложенных
организаторами тем, демонстрируя умение формулировать проблему,
анализировать источники и правильно оформлять библиографию. Обычно
конкурс проходит заочно и включает этап проверки соответствия
оформлению, экспертную оценку содержания и публичную защиту для
финалистов. Точные сроки и темы ежегодно публикуются на сайтах вузов.

\section{Рекомендации по
участию}\label{ux440ux435ux43aux43eux43cux435ux43dux434ux430ux446ux438ux438-ux43fux43e-ux443ux447ux430ux441ux442ux438ux44e}

\begin{itemize}
\tightlist
\item
  \textbf{Следите за официальными объявлениями.} Даты проведения и
  условия участия могут изменяться; актуальная информация размещается на
  сайте вашего университета и в социальных сетях студенческого научного
  общества.
\item
  \textbf{Начинайте подготовку заранее.} Даже для конкурса рефератов
  необходимо время на поиск литературы, структурирование работы и
  корректное оформление.
\item
  \textbf{Используйте ресурсы этого учебника.} Разделы по поиску
  литературы, оценке статей и оформлению библиографии помогут вам
  подготовить качественную работу.
\end{itemize}

\chapter{Дополнительные
ресурсы}\label{ux434ux43eux43fux43eux43bux43dux438ux442ux435ux43bux44cux43dux44bux435-ux440ux435ux441ux443ux440ux441ux44b}

\chapter{Обучающие платформы и статьи}\label{sec-resources}

\begin{itemize}
\item
  \href{https://medach.pro}{\textbf{Medach}} --- портал, созданный
  студентами медицинских вузов. В разделе «Как читать и писать статьи»
  есть материалы о методологии исследований, критическом мышлении и
  планировании научной работы. Полезны статьи «Как читать научные
  статьи», «Как написать первоклассную статью» и обзор «Как постигать
  медицину: фармакология».
\item
  \href{https://evidencehunt.com}{\textbf{EvidenceHunt}} --- сервис,
  позволяющий быстро находить клинические рекомендации и систематические
  обзоры. Использует искусственный интеллект для обработки запросов и
  ускоряет подготовку к обзорам.
\item
  \href{https://www.cochranelibrary.com}{\textbf{Библиотека Cochrane}}
  --- главная площадка для поиска систематических обзоров и метаанализов
  по доказательной медицине. Здесь публикуются обзоры, подготовленные
  международным сообществом исследователей.
\end{itemize}

\chapter{Курсы по статистике и
программированию}\label{ux43aux443ux440ux441ux44b-ux43fux43e-ux441ux442ux430ux442ux438ux441ux442ux438ux43aux435-ux438-ux43fux440ux43eux433ux440ux430ux43cux43cux438ux440ux43eux432ux430ux43dux438ux44e}

\begin{itemize}
\tightlist
\item
  \href{https://stepik.org/course/Основы-медицинской-статистики-1253}{\textbf{Основы
  медицинской статистики (Stepik)}} --- онлайн‑курс знакомит с базовыми
  концепциями статистики: распределениями, проверкой гипотез,
  корреляцией и регрессией.
\item
\end{itemize}

\chapter{Инструменты визуализации
данных}\label{ux438ux43dux441ux442ux440ux443ux43cux435ux43dux442ux44b-ux432ux438ux437ux443ux430ux43bux438ux437ux430ux446ux438ux438-ux434ux430ux43dux43dux44bux445}

\begin{itemize}
\tightlist
\item
  \href{https://rawgraphs.io}{\textbf{RAWGraphs}} --- открытый
  конструктор графиков. Позволяет быстро строить диаграммы и сохранять
  их в формате SVG или PNG.
\item
  \href{https://biit.cs.ut.ee/clustvis/}{\textbf{ClustVis}} ---
  веб‑сервис для кластеризации и визуализации многомерных данных. Удобен
  для создания тепловых карт и дендрограмм.
\item
  \href{https://visualizefree.com}{\textbf{VisualizeFree}} ---
  бесплатный инструмент для интерактивной визуализации данных и создания
  информационных панелей.
\end{itemize}

\chapter{Иллюстрации и
графика}\label{ux438ux43bux43bux44eux441ux442ux440ux430ux446ux438ux438-ux438-ux433ux440ux430ux444ux438ux43aux430}

\begin{itemize}
\tightlist
\item
  \href{https://www.flickr.com/photos/niaid/}{\textbf{Bioart}} ---
  коллекция изображений на биомедицинскую тематику, предоставленная
  Национальным институтом аллергии и инфекционных заболеваний США
  (NIAID). Подходит для оформления презентаций и иллюстраций.
\end{itemize}

\chapter{Советы по работе со
статьями}\label{ux441ux43eux432ux435ux442ux44b-ux43fux43e-ux440ux430ux431ux43eux442ux435-ux441ux43e-ux441ux442ux430ux442ux44cux44fux43cux438}

\begin{itemize}
\item
  \href{https://habr.com/ru/post/example/}{\textbf{Как читать научные
  статьи: советы учёных}} --- статья на Habr.com, объясняющая, как
  эффективно читать и анализировать научные публикации.
\item
  \href{https://medach.pro/post/example}{\textbf{Как оценивать научные
  статьи с помощью критического мышления}} --- материал на Medach с
  примерами вопросов, которые стоит задавать при анализе исследований.
\end{itemize}

Эти ресурсы помогут вам расширить кругозор, освоить инструменты для
исследовательской работы и повысить качество собственных публикаций.

\bookmarksetup{startatroot}

\chapter{Сообщения из
Telegram}\label{ux441ux43eux43eux431ux449ux435ux43dux438ux44f-ux438ux437-telegram}

Пересылаемые материалы из группы кружка

\hfill\break

\bookmarksetup{startatroot}

\chapter{Сообщения из Telegram}\label{sec-telegram}

\begin{figure}[H]

{\centering \includegraphics[width=2.08333in,height=\textheight,keepaspectratio]{cover.png}

}

\caption{Telegram группа кружка}

\end{figure}%

Эта страница содержит материалы, пересылаемые из группы научного кружка
в Telegram. Здесь вы найдете:

\begin{itemize}
\tightlist
\item
  \textbf{Полезные ссылки} на научные ресурсы
\item
  \textbf{Объявления} о конкурсах и мероприятиях
\item
  \textbf{Новости} из мира науки
\item
  \textbf{Советы} по научной работе
\end{itemize}

\begin{tcolorbox}[enhanced jigsaw, breakable, toprule=.15mm, colback=white, opacityback=0, rightrule=.15mm, leftrule=.75mm, arc=.35mm, bottomrule=.15mm, colframe=quarto-callout-note-color-frame, left=2mm]

\vspace{-3mm}\textbf{Группа кружка в Telegram}\vspace{3mm}

Присоединяйтесь к группе: \href{https://t.me/pharmRUM}{@pharmRUM}

Здесь вы можете задавать вопросы, делиться находками и получать
актуальную информацию.

\end{tcolorbox}

\begin{center}\rule{0.5\linewidth}{0.5pt}\end{center}

\section{Последние
материалы}\label{ux43fux43eux441ux43bux435ux434ux43dux438ux435-ux43cux430ux442ux435ux440ux438ux430ux43bux44b}

\subsection{2024 год}\label{ux433ux43eux434}

\subsubsection{Июль 2024}\label{ux438ux44eux43bux44c-2024}

\textbf{15 июля 2024} - \textbf{Новый курс по статистике} -
\href{https://stepik.org/course/Основы-медицинской-статистики-1253}{Основы
медицинской статистики на Stepik} - \textbf{Конкурс научных работ} -
Дедлайн подачи заявок 30 августа

\textbf{10 июля 2024} - \textbf{Полезная статья} -
\href{https://habr.com/ru/post/example/}{Как читать научные статьи:
советы учёных} - \textbf{Инструмент визуализации} -
\href{https://rawgraphs.io}{RAWGraphs для создания графиков}

\subsubsection{Июнь 2024}\label{ux438ux44eux43dux44c-2024}

\textbf{25 июня 2024} - \textbf{База данных} -
\href{https://evidencehunt.com}{EvidenceHunt для поиска клинических
рекомендаций} - \textbf{Образовательная платформа} -
\href{https://medach.pro}{Medach - портал для студентов-медиков}

\textbf{20 июня 2024} - \textbf{Конкурс молодых ученых} - Приглашаем к
участию студентов - \textbf{Мастер-класс} - ``Как написать научную
статью''

\begin{center}\rule{0.5\linewidth}{0.5pt}\end{center}

\section{Архив по
месяцам}\label{ux430ux440ux445ux438ux432-ux43fux43e-ux43cux435ux441ux44fux446ux430ux43c}

\subsection{2024}\label{section}

\begin{itemize}
\tightlist
\item
  \hyperref[ux438ux44eux43bux44c-2024]{Июль 2024}
\item
  \hyperref[ux438ux44eux43dux44c-2024]{Июнь 2024}
\item
  \hyperref[ux43cux430ux439-2024]{Май 2024}
\item
  \hyperref[ux430ux43fux440ux435ux43bux44c-2024]{Апрель 2024}
\end{itemize}

\subsection{2023}\label{section-1}

\begin{itemize}
\tightlist
\item
  \hyperref[ux434ux435ux43aux430ux431ux440ux44c-2023]{Декабрь 2023}
\item
  \hyperref[ux43dux43eux44fux431ux440ux44c-2023]{Ноябрь 2023}
\item
  \hyperref[ux43eux43aux442ux44fux431ux440ux44c-2023]{Октябрь 2023}
\end{itemize}

\begin{center}\rule{0.5\linewidth}{0.5pt}\end{center}

\section{Категории
материалов}\label{ux43aux430ux442ux435ux433ux43eux440ux438ux438-ux43cux430ux442ux435ux440ux438ux430ux43bux43eux432}

\subsection{📚 Образовательные
ресурсы}\label{ux43eux431ux440ux430ux437ux43eux432ux430ux442ux435ux43bux44cux43dux44bux435-ux440ux435ux441ux443ux440ux441ux44b}

\begin{itemize}
\tightlist
\item
  Курсы и лекции
\item
  Учебные материалы
\item
  Методические пособия
\end{itemize}

\subsection{🏆 Конкурсы и
мероприятия}\label{ux43aux43eux43dux43aux443ux440ux441ux44b-ux438-ux43cux435ux440ux43eux43fux440ux438ux44fux442ux438ux44f-1}

\begin{itemize}
\tightlist
\item
  Научные конкурсы
\item
  Конференции
\item
  Мастер-классы
\end{itemize}

\subsection{🔬 Научные
инструменты}\label{ux43dux430ux443ux447ux43dux44bux435-ux438ux43dux441ux442ux440ux443ux43cux435ux43dux442ux44b}

\begin{itemize}
\tightlist
\item
  Базы данных
\item
  Программы для анализа
\item
  Инструменты визуализации
\end{itemize}

\subsection{📰 Новости
науки}\label{ux43dux43eux432ux43eux441ux442ux438-ux43dux430ux443ux43aux438}

\begin{itemize}
\tightlist
\item
  Последние исследования
\item
  Открытия в медицине
\item
  Тренды в науке
\end{itemize}

\subsection{💡 Советы и
рекомендации}\label{ux441ux43eux432ux435ux442ux44b-ux438-ux440ux435ux43aux43eux43cux435ux43dux434ux430ux446ux438ux438}

\begin{itemize}
\tightlist
\item
  Как писать статьи
\item
  Методология исследований
\item
  Публикационная активность
\end{itemize}

\begin{center}\rule{0.5\linewidth}{0.5pt}\end{center}

\section{Как добавить
материал}\label{ux43aux430ux43a-ux434ux43eux431ux430ux432ux438ux442ux44c-ux43cux430ux442ux435ux440ux438ux430ux43b}

Если вы хотите поделиться полезной ссылкой или материалом:

\begin{enumerate}
\def\labelenumi{\arabic{enumi}.}
\tightlist
\item
  \textbf{Перешлите сообщение} в группу (\textbf{pharmRUM?})
\item
  \textbf{Добавьте краткое описание} к материалу
\item
  \textbf{Укажите категорию} (если возможно)
\end{enumerate}

Материалы будут добавлены на эту страницу в течение 1-2 дней.

\begin{tcolorbox}[enhanced jigsaw, breakable, toprule=.15mm, colback=white, opacityback=0, rightrule=.15mm, leftrule=.75mm, arc=.35mm, bottomrule=.15mm, colframe=quarto-callout-tip-color-frame, left=2mm]

\vspace{-3mm}\textbf{Совет}\vspace{3mm}

Подписывайтесь на группу, чтобы не пропустить важные объявления и
полезные материалы!

\end{tcolorbox}

\begin{center}\rule{0.5\linewidth}{0.5pt}\end{center}

\section{Полезные
ссылки}\label{ux43fux43eux43bux435ux437ux43dux44bux435-ux441ux441ux44bux43bux43aux438}

\begin{itemize}
\tightlist
\item
  \textbf{Группа кружка:} \href{https://t.me/pharmRUM}{@pharmRUM}
\item
  \textbf{База знаний СНК:} \url{https://svtsar.github.io/SNK/}
\item
  \textbf{Email для связи:}
  \href{mailto:sergiotsar@ya.ru}{\nolinkurl{sergiotsar@ya.ru}}
\end{itemize}

\begin{center}\rule{0.5\linewidth}{0.5pt}\end{center}

\emph{Последнее обновление:} 

\bookmarksetup{startatroot}

\chapter{Примеры
медиафайлов}\label{ux43fux440ux438ux43cux435ux440ux44b-ux43cux435ux434ux438ux430ux444ux430ux439ux43bux43eux432}

\bookmarksetup{startatroot}

\chapter{📹 Примеры
медиафайлов}\label{ux43fux440ux438ux43cux435ux440ux44b-ux43cux435ux434ux438ux430ux444ux430ux439ux43bux43eux432-1}

\section{Видео
файлы}\label{ux432ux438ux434ux435ux43e-ux444ux430ux439ux43bux44b}

\subsection{Локальное видео
(MP4)}\label{ux43bux43eux43aux430ux43bux44cux43dux43eux435-ux432ux438ux434ux435ux43e-mp4}

\subsection{YouTube видео}\label{youtube-ux432ux438ux434ux435ux43e}

\subsection{Vimeo видео}\label{vimeo-ux432ux438ux434ux435ux43e}

\section{🎵 Аудио
файлы}\label{ux430ux443ux434ux438ux43e-ux444ux430ux439ux43bux44b}

\subsection{Локальное
аудио}\label{ux43bux43eux43aux430ux43bux44cux43dux43eux435-ux430ux443ux434ux438ux43e}

\section{📄
Документы}\label{ux434ux43eux43aux443ux43cux435ux43dux442ux44b}

\subsection{PDF
документ}\label{pdf-ux434ux43eux43aux443ux43cux435ux43dux442}

\subsection{Ссылка на
PDF}\label{ux441ux441ux44bux43bux43aux430-ux43dux430-pdf}

\href{media/document.pdf}{Скачать PDF документ}

\section{🖼️
Изображения}\label{ux438ux437ux43eux431ux440ux430ux436ux435ux43dux438ux44f}

\subsection{Обычное
изображение}\label{ux43eux431ux44bux447ux43dux43eux435-ux438ux437ux43eux431ux440ux430ux436ux435ux43dux438ux435}

\begin{figure}[H]

{\centering \includegraphics[width=0.5\linewidth,height=\textheight,keepaspectratio]{media/image.jpg}

}

\caption{Описание изображения}

\end{figure}%

\subsection{Изображение с
подписью}\label{ux438ux437ux43eux431ux440ux430ux436ux435ux43dux438ux435-ux441-ux43fux43eux434ux43fux438ux441ux44cux44e}

\begin{figure}[H]

{\centering \includegraphics[width=1\linewidth,height=\textheight,keepaspectratio]{media/image.jpg}

}

\caption{Описание изображения}

\end{figure}%

Подпись к изображению

\section{📊 Интерактивные
элементы}\label{ux438ux43dux442ux435ux440ux430ux43aux442ux438ux432ux43dux44bux435-ux44dux43bux435ux43cux435ux43dux442ux44b}

\subsection{HTML5 Canvas}\label{html5-canvas}

\section{🎯 Рекомендации по размерам
файлов}\label{ux440ux435ux43aux43eux43cux435ux43dux434ux430ux446ux438ux438-ux43fux43e-ux440ux430ux437ux43cux435ux440ux430ux43c-ux444ux430ux439ux43bux43eux432}

\subsection{Для GitHub Pages
(ограничения):}\label{ux434ux43bux44f-github-pages-ux43eux433ux440ux430ux43dux438ux447ux435ux43dux438ux44f}

\begin{itemize}
\tightlist
\item
  \textbf{Видео:} до 100 МБ (рекомендуется до 50 МБ)
\item
  \textbf{Аудио:} до 25 МБ
\item
  \textbf{Изображения:} до 10 МБ
\item
  \textbf{PDF:} до 50 МБ
\end{itemize}

\subsection{Для больших файлов
используйте:}\label{ux434ux43bux44f-ux431ux43eux43bux44cux448ux438ux445-ux444ux430ux439ux43bux43eux432-ux438ux441ux43fux43eux43bux44cux437ux443ux439ux442ux435}

\begin{enumerate}
\def\labelenumi{\arabic{enumi}.}
\tightlist
\item
  \textbf{YouTube} - для видео
\item
  \textbf{Google Drive} - для документов
\item
  \textbf{Dropbox} - для любых файлов
\item
  \textbf{GitHub Releases} - для больших файлов
\end{enumerate}

\section{📝 Пример использования в
контенте}\label{ux43fux440ux438ux43cux435ux440-ux438ux441ux43fux43eux43bux44cux437ux43eux432ux430ux43dux438ux44f-ux432-ux43aux43eux43dux442ux435ux43dux442ux435}

Вот как можно вставить видео в обычный текст:

\begin{tcolorbox}[enhanced jigsaw, breakable, toprule=.15mm, colback=white, opacityback=0, rightrule=.15mm, leftrule=.75mm, arc=.35mm, bottomrule=.15mm, colframe=quarto-callout-note-color-frame, left=2mm]

\vspace{-3mm}\textbf{Пример лекции}\vspace{3mm}

Посмотрите запись лекции по фармакологии:

\textbf{Длительность:} 45 минут\\
\textbf{Тема:} Основы фармакокинетики

\end{tcolorbox}

\section{🔧 Настройка для автоматической
загрузки}\label{ux43dux430ux441ux442ux440ux43eux439ux43aux430-ux434ux43bux44f-ux430ux432ux442ux43eux43cux430ux442ux438ux447ux435ux441ux43aux43eux439-ux437ux430ux433ux440ux443ux437ux43aux438}

Добавьте в \texttt{.gitignore} исключения для медиафайлов:

\begin{verbatim}
# Разрешить медиафайлы
!media/
!media/*.mp4
!media/*.mp3
!media/*.pdf
!media/*.jpg
!media/*.png
\end{verbatim}

\section{📋 Чек-лист для добавления
медиафайлов}\label{ux447ux435ux43a-ux43bux438ux441ux442-ux434ux43bux44f-ux434ux43eux431ux430ux432ux43bux435ux43dux438ux44f-ux43cux435ux434ux438ux430ux444ux430ux439ux43bux43eux432}

\begin{itemize}
\tightlist
\item[$\square$]
  Создайте папку \texttt{media/} (если нет)
\item[$\square$]
  Поместите файл в папку \texttt{media/}
\item[$\square$]
  Добавьте HTML код в \texttt{.qmd} файл
\item[$\square$]
  Проверьте локально: \texttt{quarto\ preview}
\item[$\square$]
  Загрузите: \texttt{./auto\_deploy.sh}
\item[$\square$]
  Проверьте на сайте
\end{itemize}

\begin{center}\rule{0.5\linewidth}{0.5pt}\end{center}

\textbf{💡 Совет:} Для больших файлов лучше использовать внешние сервисы
(YouTube, Google Drive) и вставлять ссылки или iframe.




\end{document}
